\documentclass[10pt,a4paper]{article}

\usepackage[margin={2in, 0.5in}]{geometry}
\usepackage[]{graphicx,amsmath,array}

\begin{document}

% Remove Indentation
\setlength{\parindent}{0cm}

\title{Ableitungsregeln}

%redefine maketitle  to make the title it bigger
\makeatletter
\def\@maketitle{%
  \newpage
  \null
  \vskip 2em%
  \begin{center}%
  \let \footnote \thanks
    {\Huge\bfseries\@title \par}%
    \vskip 1.5em%
    {\large
      \lineskip .5em%
      \begin{tabular}[t]{c}%
        \@author
      \end{tabular}\par}%
    \vskip 1em%
    {\large \@date}%
  \end{center}%
  \par
  \vskip 1.5em}
\makeatother

\author{}
\date{}

\maketitle

\section*{Produktregel}
Die Produktregel vereinfacht das Ableiten bei Funktionen, in denen 2 Terme
miteinander Multipliziert werden, die sich als eigene Funktionen schreiben lassen.
Als Beispiel:

$f(x) = (x^2 - 1.5) \cdot (0.5x - x^2)$

Die Funktion $f(x)$ lässt sich nun aufteilen in

$u(x) = x^2 - 1.5$

$v(x) = 0.5x - x^2$

Somit lässt sich $f(x)$ als $f(x) = u(x) \cdot v(x)$ schreiben.
Die Produktregel besagt, dass bei einer Funktion dieser Art gilt

$f(x) = u(x) \cdot v(x)$

$f'(x) = u'(x) \cdot v(x) + u(x) \cdot v'(x)$

Somit ist die Ableitung der Ursprungsfunktion

$f'(x) = 2x \cdot (0.5x - x^2) + (x^2 - 1.5) \cdot (0.5 - 2x)$

\section*{Kettenregel}

Die Kettenregel ist hilfreich, wenn man eine Funktion ableiten muss, die eine Potenz enthält.
Sie zeigt, wie man eine Funktion abzuleiten kann, die als ihren Parameter eine weitere Funktion hat.

Als Beispiel:

$f(x) = (2x - 1)^3$

Die Funktion lässt sich nun in eine äußere und innere Funktion aufteilen.

$v(x) = 2x - 1$

$u(x) = x^3$

$f(x)$ lässt sich deshalb auch als $f(x) = u(v(x))$ schreiben.
Die Kettenregel besagt nun, das diese Funktion auf folgendem Weg aufgeleitet werden kann.

$f(x) = u(v(x))$

$f'(x) = v'(x) \cdot u'(v(x))$

Somit ist die Ableitung der Ursprungsfunktion

$f'(x) = 2 \cdot 3(2x - 1)^2 = 6(2x - 1)^2$

Diese setzt sich zusammen aus den Ableitungen von $v(x)$ und $u(x)$ der Ursprungsfunktion $v(x)$. 

$v'(x) = 2$

$u'(x) = 3x^2$

\section*{Quotientenregel}

\end{document}
