\documentclass[10pt,a4paper]{article}

\usepackage[margin={2in, 0.5in}]{geometry}
\usepackage[]{graphicx,amsmath,amsfonts,array}

% Remove Indentation
\setlength{\parindent}{0cm}

\begin{document}
	
\title{Übungsaufgaben zu Ableitungsregeln, AB2}

%redefine maketitle  to make the title it bigger
\makeatletter
\def\@maketitle{%
  \newpage
  \null
  \vskip 2em%
  \begin{center}%
  \let \footnote \thanks
    {\Huge\bfseries\@title \par}%
    \vskip 1.5em%
    {\large
      \lineskip .5em%
      \begin{tabular}[t]{c}%
        \@author
      \end{tabular}\par}%
    \vskip 1em%
    {\large \@date}%
  \end{center}%
  \par
  \vskip 1.5em}
\makeatother

\author{}
\date{}

\maketitle

\section*{Potenzregel}
Die wichtigste Ableitungsregel ist wohl die Potenzregel. Sie erlaubt es,
Potenzfunktionen abzuleiten wie z. B. $f(x) = x^2$.
Die formale Regel ist dabei $f(x) = x^n \rightarrow f'(x) = n \cdot x^{n-1}$.
Diese gilt sowohl für positive als auch negative Exponenten.

Beispiele für Ableitungen nach dieser Regel sind:

$f(x) = x^3 \rightarrow f'(x) = 3 \cdot x^{3-1} = 3x^2$

$f(x) = x^{-3} \rightarrow f'(x) = -3 \cdot x^{-3-1} = -3x^{-4}$

\section*{Summen- und Differenzregel}
Diese Regeln beschreibt, wie man Terme behandelt, die mit $+$ oder $-$ mit
verbunden sind.

Bei Plus und Minus, also bei $f(x) = x^2 + x^3$ oder $f(x) = x^2 - x^3$
lassen sich die Terme $x^2$ und $x^3$ als eigene Funktionen verstehen, die man einzeln
ableitet und dann wieder mit dem jeweiligen Rechenzeichen verbindet.

Für $f(x) = x^2 + x^3$ ist die Ableitung $f'(x) = 2x + 3x^2$, 

da $x^2 \rightarrow 2x$ und $x^3 \rightarrow 3x^2$.

Diese beiden Ableitungen der individuellen Funktionen, die die Gesamtfunktion
ausmachen, müssen nun nur noch mit dem Rechenzeichen verbunden werden.

Somit also auch $f(x) = x^2 - x^3 \rightarrow 2x - 3x^2$

\section*{Faktorregel}
Diese Regel beschreibt, das Faktoren vor einer Potenzfunktion, die man ableitet, erhalten bleiben.
Die Potenzfunktion $f(x) = 2x^2$ setzt sich aus dem Faktor $2$ und der Funktion $x^2$ zusammen.
Allgemein ließe sich dies auch als $f(x) = c \cdot g(x)$ beschreiben, wo $g(x)$ nun die Funktion ist,
und c der Faktor.
Bei der Ableitung der Funktion $f(x)$ bleibt der Faktor $c$ unverändert, sodass
$f'(x) = 2 \cdot 2x$.
Allgemein also $f'(x) = c \cdot g'(x)$ 

\section*{Aufgaben zu Ableitungsregeln}
Bilden sie die 1. und 2. Ableitung der Funktionen. \newline

Beispiel:

$f(x) = 2x^3$

$f'(x) = 6x^2$ 

$f''(x) = 12x$
\newline

\begin{tabular}{l l l}

	a) $f(x) = x^4$             & b) $f(x) = 2x^2$        & c) $f(x) = x^5 + 2$ \\
	d) $f(x) = 3x^3 + 2x^2 + 1$ & e) $f(x) = 1x^2 + x$    & f) $f(x) = 20x^2$ \\
	g) $f(x) = 2x^1 + 2$        & h) $f(x) = 0.5x^2 + 2x$ & i) $f(x) = 0x^3 + x^2$

\end{tabular}

\end{document}